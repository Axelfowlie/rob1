% !TEX root = rob1.tex
\chapter{Bildverarbeitung}

\section{Bildrepräsentation}
\textbf{Ein Bild ist ein 2D Gitter von diskreten Punkten (Pixel)}

\mparagraph{Koordinaten}
\begin{compactitem}
    \item $u$, horizontal
    \item $v$, vertikal
    \item Ursprung ist oben links
\end{compactitem}


\section{Farbmodelle}
\mparagraph{Graustufenbild}
Für jeden Pixel wird ein Helligkeitswert abgelegt. Normalerweise 1 Byte pro Pixel
[0,255]

\mparagraph{Monochrombild}
Diskrete Funktion
\begin{align}
    &\text{IMG}:[0..n-1] \times [0..m-1] \rightarrow [0..q]
    &(u,v) \rightarrow \text{Img}(u,v)
\end{align}

\mparagraph{Farbbild}
Verschiedene Farbmodelle für unterschiedliche Anwendungen.
\begin{compactitem}
    \item RGB
    \item HSI: geeignet für Farbsegmentierung
    \item CIE: Physikalisch
    \item CMYK: Substraktive Farbmischung
    \item YIQ
\end{compactitem}

\subsection{RGB}
\begin{compactitem}
    \item Additive Farbmischung
    \item Rot, Grün, Blau
    \item RGB24: ein Pixel durch 3 Bytes dargestellt
\end{compactitem}
\subsection{HSI}
\begin{compactitem}
    \item Hue, Saturation und Intensity/Value
    \item Trenning von Helligkeit vom Farbwert $\rightarrow$ unempfindlich gegen
    Beleuchtungsänderungen
    \item Umrechung RGB nach HSI
    \begin{compactitem}
        \item H undefiniert falls R = G = B
        \item S undefiniert falls R = G = B = 0
    \end{compactitem}
\end{compactitem}
\begin{align}
    H &= \begin{cases} \Theta, \text{ falls } B < G \\ 360 - \Theta, \text{ sonst}
        \end{cases} \\
    \Theta &= \arccos\frac{2R-G-B}{2\sqrt{(R-G)^2 + (R+B)(G-B)}} \\
    S &= 1 - \frac{3}{R + G + B}\min(R,G,B) \\
    I &= \frac{1}{3}(R+G+B)
\end{align}

\section{Lochkamera}
Projektion eines Szenenpunkt $(x,y,z)$ auf einen Bildpunkt $(u,v)$
\begin{align}
    \begin{pmatrix}u \\ v \end{pmatrix} &= \frac{f}{z} \begin{pmatrix} x \\ y \end{pmatrix} \\
    \begin{pmatrix}x \\ y \end{pmatrix} &= \frac{z}{f} \begin{pmatrix}u \\ v \end{pmatrix}
\end{align}

\section{Filteroperationen}
\subsection{Tiefpassfilter}
\textbf{Glättung, Rauschelimination}
\subsubsection{Mittelwertfilter}
\begin{displaymath}
     \begin{pmatrix}
         \frac{1}{9} & \frac{1}{9} & \frac{1}{9} \\
         \frac{1}{9} & \frac{1}{9} & \frac{1}{9} \\
         \frac{1}{9} & \frac{1}{9} & \frac{1}{9}
     \end{pmatrix}
\end{displaymath}
\subsubsection{Gauss Filter}
Ortsbereich: $ f(x,y) = \frac{1}{2\pi\sigma^2}e^{-\frac{x^2+y^2}{2\sigma^2}}$\\
Frequenzbereich:$ F(u,v) = e^{-\frac{u^2+v^2}{2}\sigma^2} $\\
Approximation von f(x) durch 3x3 Filter mit $\sigma = 0.85$
\begin{displaymath}
     F_\text{Gauss} = \frac{1}{16} \begin{pmatrix}
         1 & 2 & 1 \\ 2&4&2\\1&2&1
 \end{pmatrix}
\end{displaymath}
\subsection{Hochpassfilter}
\textbf{Kantendetektion}

\subsubsection{Prewitt}
\subsubsection{Sobel}
\subsubsection{Laplace}
\subsubsection{Roberts}
\subsection{Kombinierte Operatoren}
