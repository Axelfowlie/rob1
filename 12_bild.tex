% !TEX root = rob1.tex
\chapter{Bildverarbeitung}

\section{Bildrepräsentation}
\textbf{Ein Bild ist ein 2D Gitter von diskreten Punkten (Pixel)}

\mparagraph{Koordinaten}
\begin{compactitem}
    \item $u$, horizontal
    \item $v$, vertikal
    \item Ursprung ist oben links
\end{compactitem}


\section{Farbmodelle}
\mparagraph{Graustufenbild}
Für jeden Pixel wird ein Helligkeitswert abgelegt. Normalerweise 1 Byte pro Pixel
[0,255]

\mparagraph{Monochrombild}
Diskrete Funktion
\begin{align}
    &\text{IMG}:[0..n-1] \times [0..m-1] \rightarrow [0..q]
    &(u,v) \rightarrow \text{Img}(u,v)
\end{align}

\mparagraph{Farbbild}
Verschiedene Farbmodelle für unterschiedliche Anwendungen.
\begin{compactitem}
    \item RGB
    \item HSI: geeignet für Farbsegmentierung
    \item CIE: Physikalisch
    \item CMYK: Substraktive Farbmischung
    \item YIQ
\end{compactitem}

\subsection{RGB}
\begin{compactitem}
    \item Additive Farbmischung
    \item Rot, Grün, Blau
    \item RGB24: ein Pixel durch 3 Bytes dargestellt
\end{compactitem}
\subsection{HSI}
\begin{compactitem}
    \item Hue, Saturation und Intensity/Value
    \item Trenning von Helligkeit vom Farbwert $\rightarrow$ unempfindlich gegen
    Beleuchtungsänderungen
    \item Umrechung RGB nach HSI
    \begin{compactitem}
        \item H undefiniert falls R = G = B
        \item S undefiniert falls R = G = B = 0
    \end{compactitem}
\end{compactitem}
\begin{align}
    H &= \begin{cases} \Theta, \text{ falls } B < G \\ 360 - \Theta, \text{ sonst}
        \end{cases} \\
    \Theta &= \arccos\frac{2R-G-B}{2\sqrt{(R-G)^2 + (R+B)(G-B)}} \\
    S &= 1 - \frac{3}{R + G + B}\min(R,G,B) \\
    I &= \frac{1}{3}(R+G+B)
\end{align}

\section{Lochkamera}
Projektion eines Szenenpunkt $(x,y,z)$ auf einen Bildpunkt $(u,v)$
\begin{align}
    \begin{pmatrix}u \\ v \end{pmatrix} &= \frac{f}{z} \begin{pmatrix} x \\ y \end{pmatrix} \\
    \begin{pmatrix}x \\ y \end{pmatrix} &= \frac{z}{f} \begin{pmatrix}u \\ v \end{pmatrix}
\end{align}

\section{Filteroperationen}
\subsection{Tiefpassfilter}
\textbf{Glättung, Rauschelimination}
\subsubsection{Mittelwertfilter}
\begin{displaymath}
     \begin{pmatrix}
         \frac{1}{9} & \frac{1}{9} & \frac{1}{9} \\
         \frac{1}{9} & \frac{1}{9} & \frac{1}{9} \\
         \frac{1}{9} & \frac{1}{9} & \frac{1}{9}
     \end{pmatrix}
\end{displaymath}
\subsubsection{Gauss Filter}
Ortsbereich: $ f(x,y) = \frac{1}{2\pi\sigma^2}e^{-\frac{x^2+y^2}{2\sigma^2}}$\\
Frequenzbereich:$ F(u,v) = e^{-\frac{u^2+v^2}{2}\sigma^2} $\\
Approximation von f(x) durch 3x3 Filter mit $\sigma = 0.85$
\begin{displaymath}
     F_\text{Gauss} = \frac{1}{16} \begin{pmatrix}
         1 & 2 & 1 \\ 2&4&2\\1&2&1
 \end{pmatrix}
\end{displaymath}
Je größer $\sigma$, desto stärker die Glättung
\subsection{Hochpassfilter}
\textbf{Kantendetektion}

\subsubsection{Prewitt}
\begin{align}
    P_x &= \begin{pmatrix}-1&0&1\\-1&0&1\\-1&0&1 \end{pmatrix}
    P_y &= \begin{pmatrix}-1&-1&-1\\0&0&0\\1&1&1 \end{pmatrix}
\end{align}
Kombination der Prewitt Filter zur Bestimmung des Gradientenbetrags M
\begin{displaymath}
     M \approx \sqrt{{P_x}^2 + {P_y}^2}
\end{displaymath}
Danach Schwellwertfilterung

\section{Segmentierung}
\textbf{Aufteilung eines Bildes in aussagekräftige Segmente}
\subsection{Schwellwertfilterung}
Konvertierung eines Grauwertbildes in ein binäres Bild. \\
Generell können Objekte über ihre Farbe segmentiert werden. \\
Problematisch sind wechselnde Lichtbedinungen, Reflexionen und Schattenwürfe

\subsection{Morphologische Operatoren}
\subsubsection{Dilatation}
Vergrößert Pixel zu größeren Bereichen \\
Prüfe an jeder Position ob Maske H teilmenge von Bild B ist
\subsubsection{Erosion}
Entfernt einzelne Pixel und schwach zusammenhängende Pixelgruppen \\
Prüfe an jeder Position ob Schnittmenge von Bild B und Maske H nicht leer ist.

\subsection{Canny Kantendetektor}
\begin{compactenum}
    \item Rauschunterdrückung durch Gauss Filter
    \item Kantendetektion mit Sobel oder Prewitt
    \item Non Maximum Supression
    \begin{compactitem}
        \item Gradient muss lokales Max. sein
        \item Betrachtung der zwei direkten Nachbarn entlang der Gradientenrichtung
        \item Überprüfung erfolg gemäß dem jeweiligen Quadranten
    \end{compactitem}
    \item Hysterese Schwellwertverfahren
    \begin{compactitem}
        \item Verwende Schwellwerte low und high
        \item Liegt der Betrag des Gradienten für einen Pixel über high, so wird er
        auf jedenfall akzeptiert
        \item Ausgehen von den akzeptierten Pixel werden Kanten rekursiv verfolg
        \begin{compactitem}
            \item Überprüfung der 8 direkten Nachbarn
            \item Betrag Gradient muss über low liegen
        \end{compactitem}
    \end{compactitem}
\end{compactenum}

\section{Registrierung von Punktwolken mit Iterative Closest Point, ICP}
Registrierung ist die Zusammenführung von Punktwolken, welche das gleiche
Objekt aus unterschiedlichen Ansichten beschreibt

\subsection{Algorithmus}
Registrierung zweier 3D Punktwolken und minimiert Distanz zwischen ihnen

\begin{compactenum}
    \item Für jede Iteration k gilt:
    \begin{compactitem}
        \item Für jeden Punkt $a_i$ aus A suche Punkt $b_i$ aus B, der $a_i$ am nächsten
        liegt.
        \item Berechen eine Transformation $T_k$, so dass $D_k$ minimal wird. z.B
        $D_k = \sum_i || a_i - T_k \cdot b_i || ^2$
    \end{compactitem}
    \item $D_k$ kombiniert Translation und Rotation
    \item Wende Transformation $T_k$ auf alle Punkte aus B an
    \item Abbruchkriterium: Schwellwert für $D_{k-1}-D_k$ oder max. Iterationen erreicht.
\end{compactenum}
\begin{compactitem}
    \item \textbf{Vorteile}:
    \begin{compactitem}
        \item Algorithmus für Punkte, Normalenvektoren und andere Darstellungsformen
        anwendbar
        \item Nur einfache mathematische Operation notwendig
        \item Schnelles Registrierungsergebnis
    \end{compactitem}
    \item \textbf{Nachteile}:
    \begin{compactitem}

    \item Symmetrische Objekte können nicht ohne weiteres registriert werden
    \item Konvergenz in ein lokales Minimum möglich
    \item Überlappung der Punktwolken erforderlich
\end{compactitem}

\end{compactitem}
\section{Random Sample Consensus, RANSAC}
\begin{compactitem}
    \item \textbf{RANSAC ist ein nicht deterministischer Algorithmus. Ist eine iterative
    Methode zur Schätzung von Modellparametern aus Datenpunkten}
    \item Robust gegenüber Ausreißern und fehlenden Datenpunkten
    \item Anwendung in  Bildverarbeitung
    \begin{compactitem}
        \item Schätzung von Lininen in 2D Bildern
        \item Schätzung von Ebenen und anderen Primitiven in 3D Punktwolken
    \end{compactitem}
    \item \textbf{Vorteile}:
    \begin{compactitem}
        \item Einfach, allgemein und einfach zu implementieren
        \item Robuste Modellschätzung für Daten mit wenigen Ausreißern
        \item Vielseitig anwendbar
    \end{compactitem}
    \item \textbf{Nachteile}:
    \begin{compactitem}
        \item Nicht Deterministisch
        \item Viele Parameter
        \item Trade off zwischen Genauigkeit und Laufzeit
        \item Nicht anwendbar wenn Verhältnis Inliers/Outliers zu klein ist
    \end{compactitem}
\end{compactitem}

\subsection{Algorithmus}
\begin{compactenum}
    \item Wähle zufällig die min. Anzahl an Punkten aus, die nötig ist um die
    Modellparameter zu berechnen. (2 für Linie in 2D, 3 für Ebene in 3D)
    \item Schätze ein Modell aus dem ausgewählten Datensatz
    \item Bewertung der Modellschätzung: Berechne die Teilmenge der Datenpunkte
    (Inliers), deren Abstand zum Modell kleiner ist als ein vordefinierter Schwellwert
    \item Wiederhole 1-3 bis das Modell mit den meisten Inliers gefunden wird
\end{compactenum}

\section{Normalenschätzung in 3D Punktwolken}
\textbf{Ziel: Zusätzliche Oberflächeninformation durch Einbeziehung lokaler
Nachbarpunkte}. \\
Dient als Grundlage für weiterführende Algorithmen der Segmentierung,
Deskriptoren, Objekterkennung oder Oberflächenmodellierung

\subsection{PCA basierter Ansatz}
\begin{compactitem}
    \item Erstelle die Kovarianzmatrix C der k-Punktnachbarschaft für jeden Punkt p
    \item Bestimme die Eigenwerte und Eigenvektoren von C
    \begin{compactitem}
        \item Hauptkomponentenanalyse
        \item Eigenvektor zu kleinstem Eigenwert korrespondiert mit der Normalen
    \end{compactitem}
\end{compactitem}
\begin{displaymath}
     C = \frac{1}{k}\sum_1^k (p_i - \overline{p})(p_i - \overline{p})^T
\end{displaymath}
\begin{compactitem}
    \item $p_i$ k Nachbarpunkte
    \item $\overline{p}$ Mittelpunkt aller k Nachbarn
\end{compactitem}

\section{Simultaneous Localization and Mapping, SLAM}
\textbf{Slam ist eine Abschätzung von sowohl der Position des Roboters als auch der
Karte der Umgebung zur gleichen Zeit} \\

Slam wird benötigt um voll autonome Roboter zu bauen, um in einer unbekannten
Umgebung überleben zu können, um zu wissen wo sich der roboter befindet und zur
Navigation ohne externe Positionssysteme wie GPS

\subsubsection{Terme}
\begin{compactitem}
    \item \textbf{Localization}: Abschätzung des Aufenthaltsorts des Roboters, gegeben
    eine Karte
    \item \textbf{Mapping}: Ableiten einer Karte gegeben einem Set von Orten
    \item \textbf{SLAM}: Erstelle eine Map und lokalisiere den Roboter zeitgleich!
\end{compactitem}

\subsubsection{Problem}
Man benötigt eine Karte um den Roboter zu lokalisieren, aber zeitgleich die
Positionen des Roboters um eien Karte zu erstellen. \\
$\Rightarrow$ Tracking der Bewegung der Kamera und des Roboters

\subsubsection{Benötigte Features}
\begin{compactitem}
    \item SLAM System, dass Szenenfeatures festhält (Landmarks)
    \item Visuelle Features sind Punkte, Ecken, Linien, Texturen oder Oberflächen
    \item Features müssen beschreibend und eindeutig sein!
\end{compactitem}

\subsubsection{How to do SLAM}
\begin{compactitem}
    \item Verwende Interne Repräsentation für Position der Landmarks und der Kameraparameter
    \item Für jeden Frame (Bewege und schaue):
    \begin{compactitem}
        \item Sage hervor, um wieviel sich der Roboter bewegt hat
        \item Halte neue Landmarks fest
        \item Update die interne Repräsentation durch die Messunsicherheiten
    \end{compactitem}
    Wenn Roboter alten Landmark erkennt: Loop closing!
\end{compactitem}
