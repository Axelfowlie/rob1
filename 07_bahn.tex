% !TEX root = rob1.tex
\chapter{Bahnplanung}

\section{Problemklassen}
\begin{itemize}
    \item \textbf{Klasse a)}
    \begin{compactitem}
        \item bekannt: vollständiges Umweltmodell und vollst. Neben-, Rand- und Zwangsbedingungen
        \item gesucht: Kollisionsfreie Bahn von Start zu Zielzustand
    \end{compactitem}
    \item \textbf{Klasse b)}
    \begin{compactitem}
        \item bekannt: unvollständiges Umweltmodell unvollst. Neben-, Rand- und Zwangsbedingungen
        \item gesucht: Kollisionsfreie Bahn von Start zu Ziel
        \item Problem: Kollision mit unbekannten Objekten
    \end{compactitem}
    \item \textbf{Klasse c)}
    \begin{compactitem}
        \item bekannt: zeitinvariantes Umweltmodell (bewegliche Hindernisse)
        \item gesucht: Kollisionsfreie Bahn von Start zu Zielzustand
        \item Problem: Hindernisse in Ort und Zeit variant
    \end{compactitem}
    \item \textbf{Klasse d)}
    \begin{compactitem}
        \item bekannt: kein Umweltmodell
        \item gesucht: Kollisionsfreie Bahn von Start zu Zielzustand
        \item Problem: Kartografie
    \end{compactitem}
    \item \textbf{Klasse e)}
    \begin{compactitem}
        \item bekannt: zeitvariantes Umweltmodell
        \item gesucht: Bahn zu einem beweglichen Ziel (Rendezvous Problem)
        \item Problem: Zielzustand in Ort und Zeit beweglich
    \end{compactitem}
\end{itemize}
\section{Definitionen}
\mparagraph{Konfiguration}
Eine Konfiguration q beschreibt den Zustand eines Roboters A im eukl. Raum durch Lage und Orientierung
oder im Gelenkwinkelzustandsraum durch die Werte der Gelenke.
\mparagraph{Konfigurationsraum}
Konfigurationsraum $C$ des Roboters A ist der Raum aller möglichen Konfigurationen von A
\mparagraph{Weg}
Weg für Roboter A von der Konfiguration q$_\text{Start}$ zur Konfiguration q$_\text{Ziel}$ ist eine
stetige Abbildung $\tau:[0,1] \rightarrow C$
\mparagraph{Arbeitsraumhindernis}
Arbeitsraumhindernis H ist der Raum, welcher von einem Objekt im Arbeitsraum eingenommen wird
\mparagraph{Konfigurationsraumhindernis}
$C_H$ ist die Menge aller Punkte des Konfigurationsraum, welche zu einer Kollision mit dem Hindernis
H führen
\mparagraph{Hindernisraum}
Menge aller Konfigurationsraumhindernisse $C_\text{obst} = \cup C_{H_i}$
\mparagraph{Freiraum}
Menge aller Punkte aus C, welche nicht im Hindernisraum liegen.
$C_\text{free} =  \{ q \in C | q \notin C_\text{obst}\} = C \backslash C_\text{obst}$
