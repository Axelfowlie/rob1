% !TEX root = rob1.tex
\chapter{Bahnplanung}

\section{Problemklassen}
\begin{itemize}
    \item \textbf{Klasse a)}
    \begin{compactitem}
        \item bekannt: vollständiges Umweltmodell und vollst. Neben-, Rand- und Zwangsbedingungen
        \item gesucht: Kollisionsfreie Bahn von Start zu Zielzustand
    \end{compactitem}
    \item \textbf{Klasse b)}
    \begin{compactitem}
        \item bekannt: unvollständiges Umweltmodell unvollst. Neben-, Rand- und Zwangsbedingungen
        \item gesucht: Kollisionsfreie Bahn von Start zu Ziel
        \item Problem: Kollision mit unbekannten Objekten
    \end{compactitem}
    \item \textbf{Klasse c)}
    \begin{compactitem}
        \item bekannt: zeitinvariantes Umweltmodell (bewegliche Hindernisse)
        \item gesucht: Kollisionsfreie Bahn von Start zu Zielzustand
        \item Problem: Hindernisse in Ort und Zeit variant
    \end{compactitem}
    \item \textbf{Klasse d)}
    \begin{compactitem}
        \item bekannt: kein Umweltmodell
        \item gesucht: Kollisionsfreie Bahn von Start zu Zielzustand
        \item Problem: Kartografie
    \end{compactitem}
    \item \textbf{Klasse e)}
    \begin{compactitem}
        \item bekannt: zeitvariantes Umweltmodell
        \item gesucht: Bahn zu einem beweglichen Ziel (Rendezvous Problem)
        \item Problem: Zielzustand in Ort und Zeit beweglich
    \end{compactitem}
\end{itemize}
\section{Definitionen}
\mparagraph{Umweltmodellierung}
\begin{compactitem}
    \item \textbf{Exakt}:Beispiel für constructed solid geometry, in Form einer
    algebraischen Beschreibung
    \item \textbf{Approximiert}: Die Umwelt wird duch Näherungen beschreiben (
    Kuben, verallgem. Zylinder, Polyeder)
\end{compactitem}

\mparagraph{Planungsalgorithmen}
\begin{compactitem}
    \item \textbf{Vollständige Verfahren}: Algorithmen liefern immer einer korrekte
    Lösung und können ermitteln, ob keine Lösung existiert
    \item \textbf{Probabilistische vollständige Verfahren}: Falls eine Lösung
    existiert konvergieren die Wahrscheinlichkeiten, dass eine Lösung gefunden
    wird bei fortschreitender Zeit gegen 1. Existiert keine Lösung, kann dies nicht
    ermittelt werden.
\end{compactitem}
\mparagraph{Konfiguration}
Eine Konfiguration q beschreibt den Zustand eines Roboters A im eukl. Raum durch Lage und Orientierung
oder im Gelenkwinkelzustandsraum durch die Werte der Gelenke.
\mparagraph{Konfigurationsraum}
Konfigurationsraum $C$ des Roboters A ist der Raum aller möglichen Konfigurationen von A
\mparagraph{Weg}
Weg für Roboter A von der Konfiguration q$_\text{Start}$ zur Konfiguration q$_\text{Ziel}$ ist eine
stetige Abbildung $\tau:[0,1] \rightarrow C$
\mparagraph{Arbeitsraumhindernis}
Arbeitsraumhindernis H ist der Raum, welcher von einem Objekt im Arbeitsraum eingenommen wird
\mparagraph{Konfigurationsraumhindernis}
$C_H$ ist die Menge aller Punkte des Konfigurationsraum, welche zu einer Kollision mit dem Hindernis
H führen
\mparagraph{Hindernisraum}
Menge aller Konfigurationsraumhindernisse $C_\text{obst} = \cup C_{H_i}$
\mparagraph{Freiraum}
Menge aller Punkte aus C, welche nicht im Hindernisraum liegen.
$C_\text{free} =  \{ q \in C | q \notin C_\text{obst}\} = C \backslash C_\text{obst}$

\section{Planungsverfahren}
\subsection{Mobile Roboter}
\begin{compactitem}
    \item Voronoi Diagramm
    \item Sichtgraph $\rightarrow$ Erweiterung der Hindernisse durch Kreis/Rechteck
    \item Zellenzerlegung
    \begin{compactitem}
        \item Exakte
        \item Approximative (Quad Tree)
    \end{compactitem}
    \item Potentialfelder mit Abstoßung und Anziehungspotentialen.
\end{compactitem}
\subsection{Manipulatoren}
\subsubsection{Probabilistic Roadmaps, PRM}
\begin{compactitem}
    \item Approximation des Freiraumes
\end{compactitem}
\begin{compactenum}
    \item Vorverarbeitung
    \begin{compactitem}
        \item Zufällige Erzeugung von kollisionsfreien Stichproben
        \item Stichproben werden über kollisionsfreien Pfad verbunden
    \end{compactitem}
    \item Anfrage
    \begin{compactitem}
        \item Verbinde Start und Ziel mit Wegnetz
        \item Suche im Graphen
    \end{compactitem}
\end{compactenum}

\subsubsection{Rapidly-exploring Random Tree}
\begin{compactitem}
    \item Algorithmus zur Einmalanfrage (keine Vorverarbeitung nötig)
    \item Aufbau zweier Bäume zur Approximatin des Freiraumes
\end{compactitem}

\begin{algorithm}[H]
    \begin{algorithmic}
        \State BUILD\_RRT(K$_\text{Start},n,\epsilon$)
        \State T.init(K$_\text{Start}$) Neuer Baum mit Startkonfiguration in der Wurzel
        \For {k=1 to n}
        \State K$_\text{Zuf} \leftarrow$ RAND\_CONF() Gleichverteilt zufällige Erzeugung einer Konfiguration
        \State K$_\text{Nahe} \leftarrow$ NEAREST\_VERTEX(K$_\text{Zuf}$, T) Bestimmung des nächsten Knoten
        \State K$_\text{Neu} \leftarrow$EXTEND(K$_\text{Nahe}$, K$_\text{Zuf}$, $\epsilon$) Erzeuge neue Konfiguration mit Abstand $\epsilon$ zu K$_\text{Nahe}$
        \State Falls Konfigurations gültig, füge neuen Knoten und Kante hinzu
        \EndFor
        \caption{RRT - Einfach}
    \end{algorithmic}
\end{algorithm}

\begin{algorithm}[H]
    \begin{algorithmic}
        \State 1. Aufbau von zwei Bäumen $T_1$ und $T_2$ mit Wurzelknoten $K_\text{Start}$ bzw. $K_\text{Ziel}$
        \State 2. Stichprobe K$_\text{Zuf}$ wird zur Erweiterung von $T_1$ verwendet mit Ergebni K$_\text{Neu}$
        \State 3. Erweiterung von $T_2$ mit K$_\text{Neu}$ bis K$_\text{Neu}$ erreicht oder !Valid(K$_\text{Neu}$)
        \State 4. Falls K$_\text{Neu}$ von $T_2$ erreicht wird, ist ein gültiger Pfad gefunden
        \State 5. Sonst: Vertausche Rolle von $T_1$ und $T_2$

        \caption{RRT - Bidirektional}
    \end{algorithmic}
\end{algorithm}

\textbf{Die Wahrscheinlichkeit, dass ein Knoten erweitert wird, ist proportional zur Größe der
entsprechenden Voronoi-Region}

\subsubsection{Glättung}
\begin{compactitem}
    \item Zufällige Wahl zweier Knoten im Lösungsweg
    \item Falls Verbindung kollisionsfrei, Verbinde beide Knoten und lösche dazwischenliegende Knoten
    aus dem Lösungspfad
\end{compactitem}

\subsubsection{Constrained Motion Planning}
\begin{compactitem}
    \item RRT Erweiterung
    \item VALID(K) unzureichend für Einschränkungen mit P(VALID(K))=0
    \item Erweiterung durch Berechnung von Distanz und/oder Richtung zu Konfiguration im Freiraum
    \item Ersetzung der Extend-Funktion
    \begin{compactitem}
        \item Randomized Gradient Descend
        \item First Order Retraction
    \end{compactitem}
\end{compactitem}
