% !TEX root = rob1.tex
\chapter{Einführung}

\subsection{Begriffsbildung}

\mparagraph{Roboter}
\begin{compactitem}
    \item \textbf{Industrie}: Ein Roboter ist ein frei programmierbarer, multifunktionaler
    Manipulator mit mindenstes 3 unabhängigen Achsen, um Materialien, Teile, Werkzeuge oder
    Geräte auf programmierten, variablen Bahnen zu bewegen zur Erfüllung verschiedener Aufgaben.
    \item \textbf{Wissenschaft}: Roboter sind sensomotorische Maschinen zur Erweiterung der
    menschlichen Handlungsfähigkeit. Sie bestehen aus mechatronischen Komponenten, Sensoren und
    rechnerbasierten Kontroll- und Steuerungsfunktionen.
\end{compactitem}

\mparagraph{Robotik}
Robotik ist ein interdisziplinär ausgerichtetes Forschungsgebiet, bei dem im Mittelpunkt mechanische
Vorrichtungen und geeignete Steuereinheiten selbsttätig komplexe Aufgaben verrichten.

\subsection{Asimovsche Robotergesetze}
\begin{compactenum}
    \item Ein Roboter darf keine Menschen verletzen oder durch Untätigkeit zu Schaden kommen lassen.
    \item Ein Roboter muss den Befehlen eines Menschen gehorchen, es sei denn, solche Befehle stehen
    im Widerspruch zum ersten Gesetz
    \item Ein Robot muss seine eigene Existenz schützen, solange dieser Schutz nicht dem ersten oder
    zweiten Gesetz widerspricht.
\end{compactenum}

\subsection{Anwendungsfelder}
\mparagraph{Industrieroboter}
Ein automatisch kontrollierter, reprogrammierbarer und vielseitiger Manipulator, mit 3 oder mehr
programmierbaren Achsen, welche fix am Platz oder mobil zur industriellen automatisierten
Anwendung ist.
\mparagraph{Serviceroboter}
Roboter, der halb- oder vollautonom arbeitet, mit dem Ziel, nützliche Dienste zum Wohle von Menschen
und Einrichtung zu erledigen. Keine Aufgaben im Bereich der industriellen Produktion.
\mparagraph{,,Personal Robot''}
Roboter, der den Menschen in Sachen Bewegung, Intelligenz und Kommunikation ähnelt.
