% !TEX root = rob1.tex
\chapter{Greifplanung}

\begin{figure}[!h]
    \centering
    \includegraphics [scale=0.5]{griff}
    \caption{Cutkosky Grifftaxonomie}
\end{figure}

\section{Nebenbedinungen}
\subsection{Intern}
\begin{compactitem}
    \item \textbf{I1: Gültigkeit des Griffes}, Überlappung zwischen den Greifmerkmalen des zu greifenden
    Objektes und den Greifmerkmalen der Greiferfinger
    \item \textbf{I2: Kollisionsfreiheit des Griffes}, keine Kollision zwischen Greifer und gegriffenem
    Objekt
    \item \textbf{I3: Zugänglichkeit des Griffes}, der Griff ist für den Greifer kollisionsfrei erreichbar
\end{compactitem}
\subsection{Extern}
\begin{compactitem}
    \item \textbf{E1: Kollisionsfreie Anrückbewegung des Greifers}, keine Kollision zwischen Roboterarm,
    Grifer, benachbarten Objekten und Arbeitsebene
    \item \textbf{E2: Kollisionsfreie Abrückbewegung mit gegriffenem Objekt}, siehe e1
    \item \textbf{E3: Berücksichtigung der Roboterkinematik}, der selektierte Griff liegt im Arbeitsraum
    des Roboters und die korrespondierenden Trajektorien der Anrück und Abrückbewegung können
    können vom Roboter abgefahren werden.
    \item \textbf{E4: Stabilität des Griffes}, sowohl während der Greifbewegung der Greiferfinger als
    auch während der Transferbewegung des Greifers mit gegriffenem Objekt verändert sich die relative
    Lage und Orientierung des zu greifenden/gegriffenen Objekts zum Greifer nicht
    \item \textbf{E5: Stabilität der Szene}, Abrückbewegung des Greifers mit gegriffenem Objekt sollte
    die Stabilität der Szene nicht beeinflussen.
    \item \textbf{E6: Aufgabenabhängigkeit des Griffs}, wird Pick and Place Operation ausgeführt, dann
    muss zur Aufnahme und Ablagekonfiguration des zu greifenden Objektes kompatibler Griff gewählt werden.
    \begin{compactitem}
        \item Kann kein Griff bestimmt werden, der Nebenbedingung erfüllt, muss geeignete Umgreifsequenz
        bestimmt werden.
        \item Erfordert Griff Ausübung von Kraft und Kraftmomenten auf das Objekt, so muss Greifer diese
        Aufbringen können
    \end{compactitem}
\end{compactitem}

\section{Greifanalyse und Synthese}
\mparagraph{Greifanalyse}
\begin{compactitem}
    \item gegeben: Objekt und Menge von Kontaktpunkten
    \item gesucht: Aussagen zur Stabilität des Giffs unter Berücksichtung der Nebenbedinungen.
\end{compactitem}
\mparagraph{Greifsynthese}
\begin{compactitem}
    \item gegeben: Objekt und eine Menge von Nebenbedinungen.
    \item gesucht: Menge von Kontaktpunkten
\end{compactitem}

\section{Kontaktmodelle}
\begin{compactitem}
    \item Kontakt ohne Reibung (existiert nicht)
    \item Kontakt mit Reibung (Kraft wirkt normal als auch tangential zur Fläche)
    \item Soft-Kontakt (Kraft wirkt normal, tangential. Zusätzlich wirken axiale Momente)
\end{compactitem}
Reibungskegel wird kontinuierlich oder mit approximiert mit Polyeder beschrieben

\mparagraph{Wrenchvektor}
Vektor $\vec{w}$ fasst die in einem Kontaktpunkt $\vec{p}_i$ wirkenden Kräfte $f_i$ und Momente
$\tau_i$ mit $i \in \{x,y,z\}$ zusammen. \\
In Abhängigekeit vom Typ des i-ten Kontaktpunktes normalen \textbf{n}, tangentiale Kräft \textbf{t} und
axiale Momente \textbf{$\Theta$}: $^i\vec{w}_n, ^i\vec{w}_t, ^i\vec{w}_\Theta$ bzw. korrespondierende
Skalare $^ic_n, ^ic_t, ^ic_\Theta$
\mparagraph{Greifmatrix}
Darstellung der Wrenchvektoren für einen räumlichen Griff als Spaltenvektor einer 6 x 3m Matrix G.

\mparagraph{Gleichgewichtsgriff}
Ein Griff wird als Gleichgewichtsgriff bezeichne,t wenn die Summer aller Kräfte $f_i$ und Momente
$\tau_i$, die auf das gegriffene Objekt wirken, gleich Null ist.
\begin{align}
    &1. \forall i \in \{1, ...,m\} : ^ic_n \geq 0, ^i\mu_t * ^ic_n \geq |^ic_t|, ^i\mu_\Theta * ^ic_n
    \geq |^ic_\Theta| \\
    &2. \exists \vec{c} \in R^{3m}, \vec{c} \neq \vec{0} : G.\vec{c} + \vec{e} = 0
\end{align}
$^i\mu_t, ^i\mu_\Theta \in R$ = Coulombschen Reibungskoeffizienten am i-ten Kontaktpunkt

\section{Griffe}
\subsection{Kraftgeschlosene Griffe}
\begin{align}
    & \forall \vec{e} = {f_x, f_y, f_z, \tau_x, \tau_y, \tau_z} \in R^6 \\
    & \exists \vec{c} \in R^{3m}, \vec{c} \neq \vec{0} : G.\vec{c} + \vec{e} = 0
\end{align}
\begin{compactitem}
    \item Kraftgeschlossenheit basierend auf Punktkontakten ohne Reibung braucht mind. 4 Kontaktpunkte.
    Bei 3D höchstens 12. Wenn Objekt auf Polyeder gilt obere Grenze von 7 Kontaktpunkten.
    \item Kraftgeschlossenheit basierend auf Punktkontakten mit Reibung kann jedes planare Objekt durch
    3 Kontaktpunkte basierenden Fingerspitzen griff kraftgeschlossen gegriffen werdne. Für 3D min. 4.
\end{compactitem}
\subsection{Formgeschlosene Griffe}
\begin{align}
    & \forall \vec{e} = {f_x, f_y, f_z, \tau_x, \tau_y, \tau_z} \in R^6 \\
    & \exists \vec{c} \in R^{3m}: G.\vec{c} + \vec{e} = 0
\end{align}
\begin{compactitem}
    \item Nur von Position der Kontaktpunkte auf der Oberfläche des Objektes und den externen
    Oberflächen-Normalenvektoren abhängig.
    \item Weder Normal oder Tangentialkräfte, noch Drehmomente durch u.a Reibung werden berücksichtig.
    \item Mindestens 4 Kontaktpunkte nötig. Bei 3D min. 7.
\end{compactitem}

\begin{table}[h!]
\centering
\begin{tabular}{ll}
\textbf{Kraftschluss}                                                                                                                                                                                                                                                                           & \textbf{Formschluss}                                                                                                                                                                                                                                      \\
\begin{tabular}[c]{@{}l@{}}Kinematik der Hand kann aktive \\ Kräfte erzeugen um einer externen\\ Störung zu widerstehen\\ \\ Mit weniger Kontaktpunkten möglich. \\ Daher bei Präzisionsgriffen verwendet.\\ \\ Regelung intern auftretender Kräfte\\ bei einem Griff erforderlich\end{tabular} & \begin{tabular}[c]{@{}l@{}}Die Kontakte an sich verhindern, \\ dass sich das Objekt bewegen kann.\\ \\ Stärkere Bedingung als Kraftschluss \\ und wird oft bei Ausführung von \\ Kraftgriffen verwendet\\ \\ Analyse basierend auf Geometrie\end{tabular}
\end{tabular}
\end{table}
\subsection{Stabile Griffe}
\begin{compactitem}
    \item Einführung einer Potentialfunktion $V: R^6 \rightarrow R$. Spezifiziert die im Griff gespeicherte
    potentielle Energie in Abhängigkeit von Lage und Orientierung des gegriffenen Objekts.
    \item $\delta q= (\delta_x, \delta_y, \delta_z, \delta_\alpha, \delta_\beta, \delta_\gamma) \in
    R^6 \neq 0$ beschreibt eine infinitesimale Lageänderung des gegriffenen Objekts und $\delta V$ die daraus
    resltierende Veränderung der potentiellen Energie.
    \item Dieser Griff ist stabil falls: $\forall \delta q \in R^6: \delta V > 0$
\end{compactitem}
\section{Planung von Greifoperationen}
\begin{compactenum}
    \item Greifklassenselektion
    \item Kollisionsfreiheit der Abrückbewegung
    \item Bestimmung von Fingerhindernisse
    \item Bestimmung von Basishindernissen
    \item Zugängliche Griffe
    \item Erreichbare Griffe
    \item Griffselektion
\end{compactenum}
\section{Suchraum}
Dimension des Suchraums für den geom. und physikalischen Ansatz: \textbf{6 + n}
\begin{compactitem}
    \item 6 Paramter für Position und Orientierung der Hand im Raum
    \item n Anzahl der Konfigurationsparameter der Greiferfinger
\end{compactitem}
\section{Objektmodell}
\mparagraph{Greifobjekt}
starre, polyedrische Körper. Merkmale sind
\begin{compactitem}
    \item \textbf{Ecke}: Eckpunkte sind diejenigen Punkte eines Polyeders, an denen drei oder mehr
    Flächen zusammenlaufen
    \item \textbf{Kante}: Kanten beschreiben Liniensegmente, die durch den Schnitt zweier Flächen
    entstehen.
    \item \textbf{Fläche}: Flächen werden durch planare Polygonzüge spezifiziert, die durch die Kanten
    des Polyeders begrenzt werden.
\end{compactitem}
Modelliert in einem zusammenhängenden Graphen.
