% !TEX root = rob1.tex
\chapter{Bahnsteuerung}

\section{Grundlagen}
\mparagraph{Trajektorie}
Bewegung eines Roboters aufgefasst als Zustandsänderung über die Zeit,
relativ zu einem stationären Koordinatensystem mit Einschränkungen (Zwangsbedinungen,
Gütekriterien, Neben und Randbedinungen)

\mparagraph{Koordinatensysteme}
\begin{table}[h!]
\centering
\begin{tabular}{ll}
\textbf{Kartesischer Raum}                                                                                                                                                                                                                                                                                                                 & \textbf{Gelenkwinkelraum}                                                                                                                                                                                                                                                                                                                                             \\
\begin{tabular}[c]{@{}l@{}}Näher an der zu lösenden Aufgabe\\ \\ Angabe der Trajektorie erfolgt als
    \\ Funktion der Zustände des Roboters\\ \\ + Bahn einfacher zu formulieren\\ + Interpolation ist
     einfacher\\ - Inverse Kinematik für jeden\\ Trajektorienpunkt zu lösen\\ - Geplante Trajektorie
      nicht immer\\ ausführbar\end{tabular} & \begin{tabular}[c]{@{}l@{}}Näher an der Ansteuerung der
       Teilsysteme\\ des Roboters\\ \\ Bahnsteuerung als Funktion der Gelenkwinkelzustände\\ \\ +
       Ansteuerung der Gelenke ist einfacher\\ + Trajektorie ist eindeutig und berücksichtigt die \\
       Gelenkwinkelgrenzen\\ - Interpolation für mehrere Gelenke\\ - Formulieren der Trajektorie
       umständlicher\end{tabular}
\end{tabular}
\end{table}

\section{Interpolation}
\subsection{Punkt-zu-Punkt , PTP}
\subsection{Linear und Zirkularinterpolation}
\subsection{Splineinterpolation}
\section{Approximierte Bahnsteuerung}
\subsection{Bernsteinpolynom}
\subsection{De Casteljau Algorithmus}
