% !TEX root = rob1.tex
\chapter{Mathematische Grundlagen}

\section{Euklidischer 3D Raum}
\subsection{Basiskoordinatensystem, BKS}
3-dim. Koordinatensystem. Durch orthogonale Einheitsvektoren $\vec{e}_{X,B}, \vec{e}_{Y,B},
\vec{e}_{Z,B}$ definiert
\mparagraph{Rechtsdrehend:} $\vec{e}_{X} \times \vec{e}_{Y} = \vec{e}_{Z},\text{ }\vec{x} \times \vec{y} = \vec{z}$
\mparagraph{Linksdrehend:} $\vec{e}_{X} \times \vec{e}_{Y} = -\vec{e}_{Z},\text{ }\vec{x} \times \vec{y} = \vec{-z}$

z-Richtung und Drehrichtung mit Hilfe der Rechten Hand Regel (Daumen = z Achse)

\subsection{Definitionen}
\begin{compactitem}
    \item \textbf{Ort}: Ortsvektor vom Ursprung des BKS zum Ursprung des OKS
    \item \textbf{Orientierung}: Rotationsmatrix zur Abbildung der Einheitsvektoren des OKS auf die
    Einheitsvektoren des BKS
    \item \textbf{Lage}: Ortsvektor und Rotationsmatrx des OKS bezogen auf das BKS.
    $\vec{v} = (x,y,z,\alpha,\beta,\gamma)$
\end{compactitem}

\subsection{Freiheitsgrad und Bewegungsfreiheitsgrad}
\begin{compactitem}
    \item \textbf{Freiheitsgrad f} ist die Anzahl möglicher unabhängiger Bewegungen in Bezug auf
    das BKS. Minimale Anzahl von Translationen und Rotationen zur vollständigen Beschreibung der Lage des
    Objektes.
    \item \textbf{Bewegungsfreiheitsgrad F}: $\sum_{i}^n(F_{R_i} + F_{T_i})$
    \item F $\geq$ f
\end{compactitem}
\newpage
\section{Orientierungsbeschreibung mit 3x3 Matrix}
\subsection{Rotationsmatrizen}
\begin{align}
    R_x &= \begin{bmatrix} 1 & 0 & 0  \\ 0 &\cos \alpha& -\sin\alpha \\ 0 & \sin\alpha & \cos\alpha \end{bmatrix}\\
    R_y &= \begin{bmatrix} \cos\alpha & 0 & \sin\alpha  \\ 0 & 1 & 0 \\ -\sin\alpha & 0 & \cos\alpha \end{bmatrix}\\
    R_z &= \begin{bmatrix} \cos \alpha& -\sin\alpha& 0 \\ 0 \sin\alpha & \cos\alpha & 0 \\ 0 & 0 & 1 \end{bmatrix}
\end{align}

\subsection{Drehachse}
\begin{compactitem}
    \item \textbf{Euler Winkel}: Drehung um jeweils veränderte Achse. Jede Drehung bezieht sich auf
    das neue Koordinatensystem. Von links nach rechts
    \item \textbf{Roll, Pitch, Yaw}: Drehung um unveränderte Achse. Jede Drehung bezieht sich auf
    das BKS. Von rechts nach links
\end{compactitem}

\section{Homogene 4x4 Matrix}
\mparagraph{Rotation}
$R_x$, $R_y$, $R_z$ wie bei 3x3 nur mit $(0,0,0,1)$ Extrazeile
\mparagraph{Translation}
\begin{align}
    T_{\text{trans}} &= \begin{bmatrix}1 & 0 & 0 & 0\\0 & 1 & 0 & 0\\0 & 0 & 1 & 0\\0 & 0 & 0 & 1\end{bmatrix}
\end{align}
\mparagraph{Skalierung}
Lokal und Global
\begin{align}
    T_{\text{scale}} &= \begin{pmatrix}a & 0 & 0 & 0\\0 & b & 0 & 0\\
    0 & 0 & c & 0\\0 & 0 & 0 & 1\end{pmatrix} \begin{pmatrix}x\\y\\z\\1\end{pmatrix} = \begin{pmatrix}ax\\by\\cz\\1\end{pmatrix}
\end{align}
\begin{align}
    T_{\text{scale}} &= \begin{pmatrix}1 & 0 & 0 & 0\\0 & 1 & 0 & 0\\
    0 & 0 & 1 & 0\\0 & 0 & 0 & s\end{pmatrix} \begin{pmatrix}x\\y\\z\\1\end{pmatrix} = \begin{pmatrix}x\\y\\z\\s\end{pmatrix}
\end{align}
\subsection{Lagebeschreibung}

Beschreibung der Lage des Koordinatensystems B relativ zum Koordinatensystem A
\begin{displaymath}
    ^AH_B
\end{displaymath}
\mparagraph{Transformationsabbildung}
\begin{displaymath}
    ^AH_B: ^BP \rightarrow ^AP
\end{displaymath}
\mparagraph{Transformationsoperator}
\begin{displaymath}
    H: ^BP_1 \rightarrow ^AP_2
\end{displaymath}
\subsubsection{Verkettung}

\begin{compactitem}
    \item Lage von Objekt 1 bzgl. BKS: $^\text{BKS}H_1$
    \item Lage von Objekt 1 bzgl. Objekt 1: $^{o1}H_2$
    \item Lage von Objekt 1 bzgl. Objekt 2: $^{o2}H_3$
    \item Lage von Objekt 1 bzgl. BKS: $^\text{BKS}H_3$
\end{compactitem}

\begin{align}
     ^\text{BKS}H_3 &= ^\text{BKS}H_1 * ^{o1}H_2 * ^{o2}H_3 \\
     \text{bzw.} &\prod^n_{i=1}{}^{H_{i-1}}H_i \text{ mit } H_0 = \text{BKS}
\end{align}

\subsection{Nachteile}
\begin{compactitem}
    \item Hohe Redundanz
    \item Interpolation schwierig
\end{compactitem}

\section{Quaterionen}
\textbf{Dienen nur der Rotation, nicht der Translation!}
\mparagraph{Definition}
Ein Quaternion hat die Darstellung $q = (a,b,c,d)^T$
\begin{compactitem}
    \item $a$ ist der \textbf{Realteil}
    \item $u = (b,c,d)^T$ ist der \textbf{Imaginärteil}
\end{compactitem}
\begin{align}
    q &= a + bi + cj + dk \\
    i^2 &= j^2 = k^2 = ijk = -1 \\
    ij &= -ij = k \\
    jk &= -kj = i \\
    ik &= -ki = j
\end{align}
\begin{itemize}
    \item \textbf{Addition}: $q + r = (a_q + a_r, u_q + u_r)$
    \item \textbf{Punktprodukt/Skalarprodukt}: $q \cdot r = a_qa_r +b_qb_r + c_qc_r + d_qd_r $
    \item \textbf{Multiplikation}: $q * r = (a_q + ib_q + jc_q + kd_q) * (a_r + i_br + jc_r + kd_r)$
    \item \textbf{Konjugiertes Quaternion}: $\overline{q} = (a,-u)$
    \item \textbf{Norm}: $|q| = \sqrt{a^2+b^2+c^2+d^2}$
    \item \textbf{Multiplikatives Inverses}: $q^{-1} = \frac{\overline{q}}{|q|^2}$
\end{itemize}

\subsection{Rotation}
\begin{compactitem}
    \item Winkel $\Theta$
    \item Rotationsachse $u$
    \item zu rotierender Vektor $v$
\end{compactitem}
\begin{align}
    q &= (\frac{\cos\Theta}{2},u\frac{\sin\Theta}{2})\\
    a &= (0,v)
\end{align}
Berechne: $qa\overline{q}$
\subsection{Vorteile}
\begin{compactitem}
    \item Intuitive Darstellung von Rotationen, direkte Angabe von Drehwinkel und Achse
    \item Kompakte Darstellung
    \item Rotation um gewünschte Achse
    \item Numerische Stabilität
\end{compactitem}

\section{Duale Quaternionen}
Ersetzung der 4 Reellen werte durch Dualzahlen um Orientierung und Lage zu erhalten.
\mparagraph{Duale Zahl}
\begin{displaymath}
     d = p + \epsilon \cdot s, \text{ wobei }\epsilon^2 = 0
\end{displaymath}
mit Primärteil $p$ und Sekundärteil $s$
\mparagraph{Duales Quaternion}
\begin{displaymath}
     \text{DQ} = (d_1,d_2,d_3,d_4) \text{ mit } d_i = dp_i + \epsilon \cdot ds_i
\end{displaymath}
Der Reale Skalarteil enthält den Winkelwert $\frac{\Theta}{2}$, der imaginäre Skalarteil die
Verschiebungsgröße d. \\
Die restlichen drei Dualzahlen beschreiben eine beliebige gerichtete, normierte Gerade im Raum.
