% !TEX root = rob1.tex
\chapter{Robotermodellierung}
\section{Geometrisches Model}
\mparagraph{Einsatzbereich}
\begin{compactitem}
    \item Graphische Darstellung von Körpern
    \item Ausgangspunkt der Abstandsmessung und Kollisionserkennung
    \item Grundlage zur Berechnung der Bewegungen von Körpern
    \item Grundlage zur Ermittlung der wirkenden Kräfte und Momente.
\end{compactitem}

\mparagraph{Klassifizierung}
\begin{itemize}
    \item \textbf{Raum}: 2D, 2.5D, 3D Modelle
    \item \textbf{Grundprimitive}
    \begin{compactitem}
        \item Kanten- bzw. Drahtmodelle
        \item Flächen- bzw. Oberflächenmodelle
        \item Volumenmodell
    \end{compactitem}
\end{itemize}
\section{Kinematisches Model}
\textbf{Das kinematische Modell eines Roboters beschreibt die Zusammenhänge zwischen
dem Raum der Gelenkwinkel (Roboterkoord, Konfigurationsraum) und dem Raum der
Lage des Endeffektors in Weltkoordinaten (Arbeitsraum, Kartesischer Raum.)}
\subsection{Direktes kinematisches Problem, Vorwärtskinematik}
