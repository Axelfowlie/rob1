% !TEX root = rob1.tex
\chapter{Interaktive Programmierung}

\section{Kriterien}

\subsection{Programmierort}
\subsubsection{Direkte Programmierung}
\textbf{Programmierung erfolg direkt am Roboter}
\subsubsection{Indirekte Programmierung}
\textbf{Programmierung erfolt ohne roboter mit Hilfe textueller, grpahischer, interkativer Methoden}
\subsection{Art der Programmierung}
\subsubsection{Direkte Programmierung}
\begin{compactitem}
    \item ,,Einstellen'' des Roboters
    \item Teach-In Programmierung
    \item Playback Programmierung (Manuelle Programmierung)
    \item Master-Slave Programmierung
    \item Sensotunterstützte Programmierung
\end{compactitem}
\subsubsection{Textuelle Verfahren}
\begin{compactitem}
    \item Programierung erfolgt mittles erweitereter, höherer Programmiersprachen $\rightarrow$
    Robotersteuerungsprogramm
    \item \textbf{Vorteile}:
    \begin{compactitem}
        \item Programmierung unabhängig von Roboter
        \item Strukturierte, übersichtliche Programmierlogik
        \item Erstellung komplexer Programme
    \end{compactitem}
    \item \textbf{Nachteile}:
    \begin{compactitem}
        \item Programmierkenntnisse nötig
        \item keine / schlechte Korrekturmöglichkeiten
    \end{compactitem}
\end{compactitem}
\subsubsection{Graphische Verfahren }
\begin{compactitem}
    \item Neben textueller Beschreibung auch graphische Darstellung des Weltmodells $\rightarrow$ Simulation der
    Roboterprogramme
    \item \textbf{Vorteile}:
    \begin{compactitem}
        \item Programmierung unabhängig von Roboter
        \item wenige Programmierkenntnisse nötig
        \item einfache Programmierung, leichte Fehlererkennung
        \item schnelles Erstellen komplexer Programme
    \end{compactitem}
    \item \textbf{Nachteile}:
    \begin{compactitem}
        \item Leistungsfähige Hardwarde für Visualisierung und Simulation
        \item Komplexe Modelle für realitätsnahe Simulation nötig
        \item Roboter und Umwelt müssen modelliert werden
    \end{compactitem}
\end{compactitem}

\subsubsection{Gemische Verfahren}
\begin{compactitem}
    \item Graphische Programmierung basierend auf sensorieller Erfassung der Benutzerführung $\rightarrow$
    Simulation der Roboterprogramme
    \item \textbf{Vorteile}:
    \begin{compactitem}
        \item wenige Programmierkenntnisse nötig
        \item einfache Programmierung, leichte Fehlererkennung
        \item schnelles Erstellen komplexer Programme
    \end{compactitem}
    \item \textbf{Nachteile}:
    \begin{compactitem}
        \item sensorielle Benutzererfassung noch zu ungenau
        \item Leistungsfähige Hardware für Signalanalyse, Modellierung
        \item Komplexe Modelle nötig
    \end{compactitem}
\end{compactitem}
\subsection{Abstraktionsgrad der Programmierung}
\subsubsection{Explizite Programmierung (roboterorientiert)}
Bewegung und Greiferbefehle sind direkt in eine Programmiersprache eingebunden. \\
\textbf{,,Wie ist es zu tun?''}

\subsubsection{Implizite Programmierung (aufgabenorientiert)}
Die Aufgabe, die der Roboter ausführen soll, wird beschrieben z.B in Form von Zuständen \\
\textbf{,,Was ist zu tun?''}
\section{Interkative Programmierverfahren}
