% !TEX root = rob1.tex
\chapter{Interaktive Programmierung}

\section{Kriterien}

\subsection{Programmierort}
\subsubsection{Direkte Programmierung}
Programmierung erfolg direkt am Roboter
\subsubsection{Indirekte Programmierung}
Programmierung erfolt ohne Roboter mit Hilfe textueller, graphischer, interaktiver Methoden
\subsection{Art der Programmierung}
\subsubsection{Direkte Programmierung}
\begin{compactitem}
    \item ,,Einstellen'' des Roboters
    \item Teach-In Programmierung
    \item Playback Programmierung (Manuelle Programmierung)
    \item Master-Slave Programmierung
    \item Sensorunterstützte Programmierung
\end{compactitem}
\subsubsection{Textuelle Verfahren}
\begin{compactitem}
    \item Programierung erfolgt mittles erweiterter, höherer Programmiersprachen $\rightarrow$
    Robotersteuerungsprogramm
    \item \textbf{Vorteile}:
    \begin{compactitem}
        \item Programmierung unabhängig von Roboter
        \item Strukturierte, übersichtliche Programmierlogik
        \item Erstellung komplexer Programme
    \end{compactitem}
    \item \textbf{Nachteile}:
    \begin{compactitem}
        \item Programmierkenntnisse nötig
        \item keine / schlechte Korrekturmöglichkeiten
    \end{compactitem}
\end{compactitem}
\subsubsection{Graphische Verfahren }
\begin{compactitem}
    \item Neben textueller Beschreibung auch graphische Darstellung des Weltmodells $\rightarrow$ Simulation der
    Roboterprogramme
    \item \textbf{Vorteile}:
    \begin{compactitem}
        \item Programmierung unabhängig von Roboter
        \item wenige Programmierkenntnisse nötig
        \item einfache Programmierung, leichte Fehlererkennung
        \item schnelles Erstellen komplexer Programme
    \end{compactitem}
    \item \textbf{Nachteile}:
    \begin{compactitem}
        \item Leistungsfähige Hardware für Visualisierung und Simulation
        \item Komplexe Modelle für realitätsnahe Simulation nötig
        \item Roboter und Umwelt müssen modelliert werden
    \end{compactitem}
\end{compactitem}

\subsubsection{Gemische Verfahren}
\begin{compactitem}
    \item Graphische Programmierung basierend auf sensorieller Erfassung der Benutzerführung $\rightarrow$
    Simulation der Roboterprogramme
    \item \textbf{Vorteile}:
    \begin{compactitem}
        \item wenige Programmierkenntnisse nötig
        \item einfache Programmierung, leichte Fehlererkennung
        \item schnelles Erstellen komplexer Programme
    \end{compactitem}
    \item \textbf{Nachteile}:
    \begin{compactitem}
        \item sensorielle Benutzererfassung noch zu ungenau
        \item Leistungsfähige Hardware für Signalanalyse, Modellierung
        \item Komplexe Modelle nötig
    \end{compactitem}
\end{compactitem}
\subsection{Abstraktionsgrad der Programmierung}
\subsubsection{Explizite Programmierung (roboterorientiert)}
Bewegung und Greiferbefehle sind direkt in eine Programmiersprache eingebunden. \\
\textbf{,,Wie ist es zu tun?''}

\subsubsection{Implizite Programmierung (aufgabenorientiert)}
Die Aufgabe, die der Roboter ausführen soll, wird beschrieben z.B in Form von Zuständen \\
\textbf{,,Was ist zu tun?''}
\section{Interaktive Programmierverfahren}

\subsection{Grundprinzip und Komponenten}
\begin{compactitem}
    \item Mensch ist Domänenexperte (Manipulation) (\textbf{Benutzer})
    \item Explizite Demonstration als Manipulationsaufgabe (\textbf{Benutzer})
    \item Sensorielle Erfassung der Demonstration (\textbf{Sensoren und Interaktionsformen})
    \item Erzeugung der internen Repräsentation des Roboterprogramms (\textbf{Lernsystem})
    \item Abbildung auf das Robotersystem (\textbf{Transformation})
    \item \textbf{Ausführung}
\end{compactitem}

\mparagraph{Interaktionsformen}
\begin{compactitem}
    \item Physische Demonstration
    \item Graphische Demonstration
    \item Ikonische Demonstration
    \item Kommentierung
\end{compactitem}
\mparagraph{Sensoren}
\begin{compactitem}
    \item fixierte Deckenkamera
    \item aktiver Kamerakopf
    \item Taktile Sensoren im Handbereich
    \item Trackingsystem
    \item Datenhandschuh
\end{compactitem}

\mparagraph{Lernsystem, Wissensrepräsentation}
\begin{compactitem}
    \item Manipulatorabhängige Repräsentation durch Angabe von Aktionssequenzen oder Gelenkwinkel-,
    Kraft und Momenttrajektorie
    \item Manipulatorunabhängige Repräsentation durch Sequenzen von Elementaroperatoren
    \begin{compactitem}
        \item Elementaroperatoren sind Regelungen mit Start-, End- und Fehlerkriterien
        \item Implementierung der Elementarop. ist manipulatorunabhängig
        \item Effekte in der Umwelt sind manipulatorunabhängig.
    \end{compactitem}
\end{compactitem}

\mparagraph{Transformation}
Abbildung des generalisierten Handlungswissens auf ein Robotersystem in der Ausführungsumgebung.
\subsection{Probabilistisches Programmierverfahren}
\begin{compactitem}
    \item Lernen von Skills durch Aktive, physische Demonstration am Roboter.
    \item Repräsentation durch GMMs
    \item Direkte Ausführung
    \item \textbf{Vorteile}:
    \begin{compactitem}
        \item schnelles Verfahren
        \item automatische Adaptierung an Änderung der Objektposition
    \end{compactitem}
    \item \textbf{Nachteile}:
    \begin{compactitem}
        \item relevante Merkmale manuell definiert
        \item geringe Generalisierung, da keine Vorbedingung, Ziele, Kollisionen
        \item keine Validierung
    \end{compactitem}
\end{compactitem}

\subsection{Dynamikbasiertes Programmierverfahren}
\begin{compactitem}
    \item Lernen von Skills durch aktive physische Demonstration am Roboter
    \item Repräsentation durch Dznamic Movement Primitives (Differentialgleichungen)
    \item Direkte Ausführung
    \item \textbf{Vorteile}:
    \begin{compactitem}
        \item schnelles Verfahren
        \item autom. Adaptierung an Start und Ziel
        \item lokale Hindernisvermeidung möglich
    \end{compactitem}
    \item \textbf{Nachteile}:
    \begin{compactitem}
        \item relevante Merkmale manuell definiert
        \item geringe Generalisierung
        \item keine Validierung
    \end{compactitem}
\end{compactitem}
\subsection{Planungsbasiertes Programmierverfahren}
\textbf{Einsatz von Planungsmethoden}
\begin{compactitem}
    \item Repräsentation der Manipulationsaufgabe als Bahnplanungsproblem mit Einschränkungen
    \item Autonome Planung von Bewegung, die das Ziel einer Manipulationsaufgabe erfüllen
    \item \textbf{Vorteile}:
    \begin{compactitem}
        \item Generalisierung auf Basis von Koordinatensystemen
        \item Start und Zielbeschreibung, Validierbarkeit
        \item Hindernisvermeidung und Berücksichtigung von Einschränkungen
        \item mehrere Lösungen und beliebige Optimalitätskriterien
    \end{compactitem}
    \item \textbf{Nachteile}:
    \begin{compactitem}
        \item hoher Aufwand (Planungszeit, Simulationszeit)
        \item 3D Modelle der Objekte, menschl. Hand notwendig
        \item autom. Segmentierung bei dynamischer Bewegung schwierig
    \end{compactitem}
\end{compactitem}
