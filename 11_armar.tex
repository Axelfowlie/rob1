% !TEX root = rob1.tex
\chapter{ARMAR}

 \section{Greifen}
 \subsection{Offline Grasp Analyse}
 \begin{compactitem}
     \item Durchführbahre Griffe werden für jedes Objekt offline berechnet und
     zusammen mit dem Objektmodell gespeichert
     \item Griff ist definiert durch
     \begin{compactitem}
         \item Grasping Center Point des Objekts der mit dem TCP auf einer Linie
         ist
         \item Approach Vector, beschreibt den Winkel in dem die Hand den Greifpunkt
         anfährt
         \item Wrist Orientation
         \item Initial Finger Configuration
     \end{compactitem}
 \end{compactitem}

 \subsection{Objekt Klassen}
 \subsubsection{Bekannte Objekte}
 \begin{compactitem}
     \item Objektgeometrie bekannt
     \item Verwende bekannte Greifplanungsmethoden für bekannte Objekte
     \item Hard
 \end{compactitem}
 \subsubsection{Verwandte Objekte}
 \begin{compactitem}
     \item Klasse des Objekts ist bekannt, z.B Flasche
     \item Wiederverwendung von Greifwissen für bekannte Klassenmitglieder auf
     neues Objekt
     \item Harder
 \end{compactitem}
 \subsubsection{Unbekannte Objekte}
 \begin{compactitem}
     \item Kein Wissen über Objekt vorhanden
     \item Herausforderung: Arbeiten mit (unvollständigen) Sensordaten (Stereovision,
     RGB-D, Laser Scan, Haptische Daten...), Trennung/Segentierung vom Hintergrund und
     Bau eines (Teil)Objektmodells
     \item Hardest!
 \end{compactitem}

 \subsection{Grasping by parts / part-based grasping}
 \begin{compactitem}
     \item Idee ist Objekte in einfache Teile aufzulösen.
     \item Plane Griffe für einzelne Segmente
 \end{compactitem}
 Forward-Ansatz der Planung
 \begin{compactitem}
     \item Führe Hand an Objekt bis Kollision
     \item Schließe finger um Objekt bis Kollision
     \item Evaluiere Kontakt von Hand und Objekt. Ist der Griff ,,force-closure''
     oder nicht
 \end{compactitem}


 \section{Box Decomposition Algorithm}
 \subsection{Von Punkten zu Boxen}
 \begin{compactitem}
     \item Nimm eine Punktwolke und bestimte Minimum Volume Bounding Box mit
     $O(n log n + n/\epsilon^3)$
     \item Es wird ein Aufspaltungs Kriterium benötigt, welches die haupt MVBB in
     eine menge kleinere Boxen aufteilt, welche die Form detaillierter darstellen
     \item Teile Form in semantische Teile auf z.B Kopf und Körper
     \item Benutze Flächen, die parallel zur Eltern MVBB ist
     \item Ein Split einer Parentbox ist besser, je weniger Volumen die resultierenden
     Kinder beinhalten
 \end{compactitem}

\subsection{Von Boxen zu Griffen}
\begin{compactitem}
    \item Jede Box hat 6 Seiten welche ein Set von intuitiven Greifhypothesen liefern:
    Entlang dem Zentrum, oder entlang der Normale
    \item Die Facedimension kann benutzt werden, umm Greifhypothesen abzuweisen, die
    größer als die Handöffnung sind
    \item Für jede Seite kann ohne weiters 4 Hypothesen, welche sich in der Handorientierung
    unterscheiden, erzeugen
    \item Blockierte oder verdeckte Seiten können einfach ausgeschlossen werden durch
    geometrische Berechnungen
\end{compactitem}

\section{Medial Axis}
\begin{compactitem}
    \item Repräsentieren das topologische Skelet eines Objektes
    \item Die Medial Axis ist eine Formapproximation eines Objekte mit Hilfe von
     Sphären maximalen Durchmessers.
    \item Sphären müssen die Geometrische Form von innen an min. 2 Punkten berühren
    \item Medial Axsis ist als Vereinigung aller Sphärenzentren definiert

\end{compactitem}
\mparagraph{Vorteile}
\begin{compactitem}
    \item Gute Approximation der Objektgeometrie
    \item Details bleiben erhalten
    \item Symmetrische Eigenschaften eines Objektes können aus den Medial Axis
    extrahiert werden
    \item Einfach um Greifplanung zu evaluieren
\end{compactitem}

\subsection{Algorithmus zur identifizierung von Slices einer Medial Axis}
\begin{compactenum}
    \item Minimum Spanning Tree der Sphären Zentren in der Ebene
    \item Lösche Kanten die länger sind als Schwellwert
    \item Clustere die entstandenen Bäume und führe für jedes Cluster aus:
    \begin{compactitem}
        \item Bestimme Volumen der Konvexen Hülle
        \item Bestimme Randpunkte der Konvexen Hülle
        \item Bestimmte Zweigknoten (Knoten mit Anzahl Kanten > 2)
        \item Bestimmte Zentrum der Gravity
    \end{compactitem}
\end{compactenum}
